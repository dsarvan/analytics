\documentclass[a4paper,11pt,openright]{report}
\setlength{\parindent}{0pt} % set noindent for entire file

\usepackage[utf8]{inputenc}
\usepackage[a4paper, top=2cm, left=1cm, right=1.5cm]{geometry}
\usepackage{xcolor,graphicx}
\usepackage{amsmath}
\usepackage{setspace}
\usepackage{sectsty}
\usepackage{etoolbox}
\usepackage{enumitem}
\usepackage{listings}
\usepackage{times}

\graphicspath{ {/home/saran/Analytics/May_01/} }

\lstdefinestyle{mystyle}{
	backgroundcolor=\color{white},
	basicstyle=\ttfamily\footnotesize,
	breakatwhitespace=false,
	breaklines=true,
	captionpos=b,
	keepspaces=true,
	showspaces=false,
	showstringspaces=false,
	showtabs=false,
	tabsize=4
}

\lstset{style=mystyle}

\begin{document}
\singlespacing
\pagestyle{plain}

\begin{center}
\textbf{Assignment Binomial Distribution} \\
Date: 04/05/2020 \hspace{2mm} Name: D.Saravanan
\end{center}

\vspace{10px}

\begin{enumerate}

\item[1.] For a Binomial Distribution parameter $n = 5$ and $p = 0.3$ Find the probabilities
of getting

\begin{enumerate}

\item[a)] At least $3$ successes \\
The probability that a random variable $X$  with binomial distribution $B(n,p)$ is equal to
the value $k$, where $k = 0, 1,...n$, is given by

\begin{equation*}
P(X = k) = \binom nk p^{k} (1-p)^{n-k} = \frac{n!}{k! (n-k)!} p^{k} (1-p)^{n-k}
\end{equation*}

\begin{equation*}
\begin{split}
		P(X \geq 3) & = P(X = 3) + P(X = 4) + P(X = 5) \\ \\
		& = \binom 53 (0.3)^{3} (1-0.3)^{2} + \binom 54 (0.3)^{4} (1-0.3) + \binom 55 (0.3)^{5} \\ \\
		& = \frac{5!}{3! (5-3)!} (0.3)^{3} (0.7)^{2} + \frac{5!}{4! (5-4)!} (0.3)^{4} (0.7) + \frac{5!}{5! (5-5)!} (0.3)^{5}
\end{split}
\end{equation*}

\item[b)] At most $3$ successes \\
\begin{equation*}
\begin{split}
		P(X \leq 3) & = P(X = 0) + P(X = 1) + P(X = 2) + P(X = 3) \\ \\
		& = \binom 50
\end{split}
\end{equation*}

\item[c)] Exactly $3$ failures \\

\end{enumerate}

\item[2.] If on an average one vessel in every ten is wrecked, find the probability that out
of five vessels expected to arrive, four at least will arrive safely. 

\begin{enumerate}

\item[a)] Multiple of 5 \\
The sample space is S = \{1, 2, 3, . . ., 100\} \\
From integers 1 to 100, there are 20 integers that are multiple by 5, which are known from
$100//5 = 20$. \\
Let event E = multiple of 5. 
\begin{equation*}
P(E) = \frac{20}{100} = 0.2
\end{equation*}

\item[b)] Divisible by 7 \\
From integers 1 to 100, there are 14 integers that are divisible by 7, which are known from
as $100//7 = 14$. \\
Let event E = divisible by 7.
\begin{equation*}
P(E) = \frac{14}{100} = 0.14
\end{equation*}

\item[c)] Greater then 70 \\
From integers 1 to 100, there are 30 integers that are greater than 70. \\
Let event E = greater then 70. \\
\begin{equation*}
P(E) = \frac{30}{100} = 0.3
\end{equation*}

\end{enumerate}

\item[3.] Five coins are tossed 3,200 times.

\begin{enumerate}

\item[a)] Find the Frequencies of the distribution of heads and tabulate the results.
\item[b)] Calculate the mean number of sucess and standard deviations.

\end{enumerate}

\end{enumerate}
\end{document}
