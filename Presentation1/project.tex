\documentclass[8pt]{beamer}
\mode<presentation>{}

\usepackage{times}
\usepackage{amsmath}
\usepackage{amsfonts}
\usepackage{amssymb}
\usepackage{graphicx}
\usepackage{booktabs}
\usepackage{xspace}
%\usepackage{caption}
%\usepackage{subcaption}

\graphicspath{{/home/ndayalan/Analytics/Presentation1/Images/}}
\usetheme{metropolis}
\setbeamercolor{frametitle}{fg=mDarkTeal, bg=black!2}

\title{Big Data and Physics}
\author{\normalsize{D. Saravanan}}
\institute{
	\normalsize{Advance Diploma in Big Data Analytics}\\
	\normalsize{National Institute of Electronis and Information Technology Chennai}
	}
\date{02/03/2020}

\begin{document}

\begin{frame}
\titlepage
\end{frame}

\begin{frame}
\frametitle{Outline}
\begin{itemize}
\item Introduction
\item Physics preamble
\item LHC Experiment
\item Calculation of Invariant Mass
\end{itemize}
\end{frame}

\begin{frame}
\frametitle{Introduction}
\begin{itemize}
\begin{figure}[ht!]
\includegraphics[width=0.7\linewidth]{lhc_computing.jpeg}
\end{figure}
\item In the era of big data, every scientific discipline must find a way to tackle
 challenges in storing, handling and interpreting large amounts of raw information.
 \\[0.5cm]
\item As scholarly research is becoming increasingly digitized, the importance of 
 managing and sharing data is being felt throughout the scientific community.\\[0.5cm]
\item As data, software and ideas become available to everyone, science can take 
advantage of the network effect to radically accelerate.\\[0.5cm]
\end{itemize}
\end{frame}

\begin{frame}
\frametitle{Physics preamble}
\begin{figure}[ht!]
\includegraphics[width=0.7\linewidth]{scale1.jpeg}
\centering
\end{figure}
\begin{itemize}
\item If the protons and the neutrons were 10 cm across, then the quarks and electrons
would be less than 0.01 mm in size and the entire atom would be about 10 km
across.\\[0.5cm]
\item No particle can move with a speed faster than the speed of light in vacuum, 
however, there is no limit to the energy a particle can attain.\\[0.5cm]
\item Energy and mass are two sides of the same coin. Mass can transform into energy
and vice versa in accordance with Einstein's equation $(E = mc^{2})$.\\[0.5cm]
\item In high-energy accelerators, particles normally travel very close to the speed of
light and the transformation happens at each collision.
\end{itemize}
\end{frame}

\begin{frame}
\frametitle{Physics preamble}
\begin{itemize}
\item Gravitation: Gravity makes apples fall to the ground. It is an attractive force.
On an astronomical scale it binds matter in planets and stars, and holds stars together
in galaxies.
[Felt by: all particles; Carrier: Graviton (Not yet discovered)]\\[0.5cm]
\item Electromagnetic: It holds electrons to nuclei in atoms, binds atoms into molecules
and is responsible for the properties of solids, liquids and gases.
[Felt by: quarks and charged leptons; Carrier: photons]\\[0.5cm]
\item Strong: The strong force binds quarks together to make protons and neutrons (and
other particles). It also binds protons and neutrons in nuclei, where it overcomes the
enormous electrical repulsion between protons. Particles that interact via the strong
force are also called 'hardons'.
[Felt by: quarks; Carrier: gluons]\\[0.5cm]
\item Weak: The weak force underlies natural radioactivity. It is also essential for the
nuclear reactions in the centres of stars like the Sun, where hydrogen is converted into
helium.
[Felt by: quarks and leptons; Carrier: intermediate vector bosons] 
\end{itemize}
\end{frame}

\begin{frame}
\frametitle{Physics preamble}
\begin{figure}[ht!]
\includegraphics[width=0.7\linewidth]{smep.png}
\centering
\end{figure}
\end{frame}

\begin{frame}
\frametitle{LHC Experiment}
\begin{itemize}
\begin{figure}[ht!]
\includegraphics[width=0.7\linewidth]{detector.png}
\end{figure}
\item \textbf{L}arge \textbf{H}ardon \textbf{C}ollider accelerator at CERN in Geneva.
\\[0.5cm]
\item At the LHC, two beams of 100 billion subatomic particles are collided at high
speeds in the hopes of finding Higgs boson.\\[0.5cm]
\item One of the main challenges in running the LHC is handling the vast amounts of data
it produces.\\[0.5cm]
\end{itemize}
\end{frame}

\begin{frame}
\frametitle{LHC Experiment}
\begin{figure}[ht!]
\includegraphics[width=0.7\linewidth]{collider.png}
\end{figure}
\end{frame}

\begin{frame}
\frametitle{LHC Experiment}
\begin{figure}[ht!]
\includegraphics[width=0.7\linewidth]{collision.jpeg}
\end{figure}
\end{frame}

\begin{frame}
	\frametitle{LHC Experiment}
	\begin{figure}[ht!]
		\includegraphics[width=0.7\linewidth]{collisions2.png}
	\end{figure}
\end{frame}

%\begin{frame}
%\frametitle{LHC Experiment}
%\begin{itemize}
%\item We get protons by stripping electrons from hydrogen atoms.\\[0.5cm]
%\item Protons are injected into the PS Booster (PSB) at an energy of 50 MeV.\[0.5cm]
%\item The booster accelerates them to 1.4 GeV. The beam is then fed to the Proton
%Synchrotron (PS) where it is accelerated to 25 GeV.\\[0.5cm]
%\item Protons are then sent to the Super Proton Synchroton (SPS) where they are 
%accelerated to 450 GeV.\\[0.5cm]
%\item They are finally transferred to the LHC (both in a clockwise and an anticlockwise
%direction) where they are accelerated for 20 minutes to 6.5 TeV. Beams circulate for
%many hours inside the LHC beam pipes under normal operating conditions.\\[0.5cm]
%\end{itemize}
%\end{frame}

\begin{frame}
\frametitle{LHC Experiment}
\begin{itemize}
\item Lets compare the LHC to a 100-megapixel digital camera that takes 40 million 
electronic "pictures" of the colliding proton bunches per second.\\[0.5cm] 
\item To keep the amount of data within reason, "empty" pictures - pictures that contain
no data - are immediately thrown away.\\[0.5cm]
\item The challenge researches are facing is to keep the interesting "pictures" for
further analysis and filter and throw away the ones that are empty.\\[0.5cm]
\item With 40 million pictures per second taken by the Detector, the experiment produces
40 TB of raw data per second, which are being filtered down to 1GB per second.\\[0.5cm]
\item To further complicate the challenge, only one picture in 100 billion is, for
example, a Higgs boson. 
\end{itemize}
\end{frame}

\begin{frame}
\frametitle{Data Set information}
\begin{itemize}
\item Data analysis in Bash using the CMS Open Data from 2011.\\[0.5cm]
\item Datasets derived from the Run2011A SingleElectron, SingleMu,
DoubleElectron, and DoubleMu primary datasets.\\[0.5cm]  
\item These data were selected from the primary datasets in order to obtain candidate
J/psi and Y events, candidate W and Z boson events and general di-electron and dimuon 
spectra.\\[0.5cm]
\item Dataset semantics:
\begin{table}[htbp]
\begin{center}
\begin{tabular}{l l l}
Run & : & The run number of the event\\
Event & : & The event number\\
E,px,py,pz & : & For a lepton, either a muon or an electron, the total\\
	   &   & energy and components of the momentum (in GeV)\\
pt         & : & The transverse momentum of the lepton (in units of GeV),\\
	   &   & either a muon or an electron\\
eta       & : & The pseudorapidity of the lepton, either a muon or an electron\\
phi   & : & The phi angle (in radians) of the lepton, either a muon or an electron\\
Q     & : & The charge of the lepton, either a muon or an electron\\
M     & : & The invariant mass (in GeV) of either two muons or two electrons\\
\end{tabular}
\end{center}
\end{table}
\end{itemize}
\end{frame}

\begin{frame}
\frametitle{Calculating the Invariant Mass}
\begin{itemize}
\item The CSV file contains events from the CMS detector where two muons were detected.
\item The value for the mass is calculated from the values of the energy and the
momentum.
\item The invariant mass can be calculated with the following equation:
	\begin{equation*}
		M = \sqrt{(\sum E)^{2} - || \sum p ||^{2} }
	\end{equation}
where $M$ is the invariant mass, $\sum E$ is the total energy, and $\sum p$ is the total
momentum. To calculate the invariant mass in the code, we are going to use the values of
$px, py, pz$ and the values of the energy for the two particles.
\end{itemize}
\end{frame}

\begin{frame}
\frametitle{Results}
\begin{figure}[ht!]
\includegraphics[width=1\linewidth]{jpsi.pdf}
\end{figure}
\end{frame}

\begin{frame}
\frametitle{Results}
\begin{itemize}
\item Now in the histogram we can clearly see a very big peak around 3.1 GeV and a 
smaller peak around 3.7 GeV.\\[0.5cm]
\item These two peaks correspond to the mass of two particles that have di-muon decay
(decay into a muon and an anti-muon).\\[0.5cm]
\item If we look in the Particle Data Group database, we find that these two particles
are mesons: the J/psi(1S) meson and the psi(2S) meson, respectively.
\end{itemize}
\end{frame}

%\begin{frame}
%\frametitle{LHC Experiment}
%\begin{figure}
%\centering
%\begin{subfigure}{0.5\textwidth}
%	\centering
%	\includegraphics[width=0.3\linewidth]{enlert.jpg}
%	\caption{Francois Englert}
%	\label{fig:sub1}
%\end{subfigure}%
%\begin{subfigure}(0.5\textwidth}
%	\centering
%	\includegraphics[width=0.3\linewidth]{higgs.jpg}
%	\caption{Peter W. Higgs}
%	\label{fig.sub2}
%\end{subfigure}
%\end{figure}
%\end{frame}
 
\end{document}
