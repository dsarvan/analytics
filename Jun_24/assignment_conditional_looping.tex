\documentclass[a4paper,11pt,openright]{report}
\setlength{\parindent}{0pt} % set noindent for entire file

\usepackage[utf8]{inputenc}
\usepackage[a4paper,top=20mm,bottom=25mm,left=10mm,right=10mm]{geometry}
\usepackage{xcolor,graphicx}
\usepackage{amsmath}
\usepackage{setspace}
\usepackage{sectsty}
\usepackage{etoolbox}
\usepackage{enumitem}
\usepackage{listings}
\usepackage{textcmds}
\usepackage{times}

\graphicspath{ {/home/saran/Analytics/Jun_24/} }

\lstdefinestyle{mystyle}{
	backgroundcolor=\color{white},
	basicstyle=\ttfamily\footnotesize,
	breakatwhitespace=false,
	breaklines=true,
	captionpos=b,
	keepspaces=true,
	showspaces=false,
	showstringspaces=false,
	showtabs=false,
	tabsize=4
}

\lstset{style=mystyle}

\begin{document}
\singlespacing
\pagestyle{plain}

\begin{center}
\textbf{Assignment Conditional and Looping Statements} \\
Date: 24/06/2020 \hspace{2mm} Name: D.Saravanan
\end{center}

\vspace{10px}

\begin{enumerate}

\item[1.] Write a Java program to display Fibonacci primes.

Program:
\lstinputlisting[language=Java]{fprime.java}
Output:
\lstinputlisting{output1.txt}

\pagebreak

\item[2.] Write a Java program to get \q{n} numbers from user and display the a) Minimum
value, b) Maximum value, c) Mean of all the entered values.

Program:
\lstinputlisting[language=Java]{value.java}

\vspace{0.5cm}

Output:
\lstinputlisting{output2.txt}

\end{enumerate}
\end{document}
