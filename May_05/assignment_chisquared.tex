\documentclass[a4paper,11pt,openright]{report}
\setlength{\parindent}{0pt} % set noindent for entire file

\usepackage[utf8]{inputenc}
\usepackage[a4paper,top=20mm,bottom=25mm,left=15mm,right=20mm]{geometry}
\usepackage{xcolor,graphicx}
\usepackage{amsmath}
\usepackage{setspace}
\usepackage{sectsty}
\usepackage{etoolbox}
\usepackage{enumitem}
\usepackage{listings}
\usepackage{textcmds}
\usepackage{times}

\graphicspath{ {/home/saran/Analytics/May_05/} }

\lstdefinestyle{mystyle}{
	backgroundcolor=\color{white},
	basicstyle=\ttfamily\footnotesize,
	breakatwhitespace=false,
	breaklines=true,
	captionpos=b,
	keepspaces=true,
	showspaces=false,
	showstringspaces=false,
	showtabs=false,
	tabsize=4
}

\lstset{style=mystyle}

\begin{document}
\singlespacing
\pagestyle{plain}

\begin{center}
\textbf{Assignment Chi-Squared test} \\
Date: 05/05/2020 \hspace{2mm} Name: D.Saravanan
\end{center}

\vspace{10px}

\begin{enumerate}

\item[1.] A random sample of employees of a large company was selected and the employees
were asked to complete a questionnaire. One question asked was whether the employee was in
favour of the introduction of flexible working hours. The following table classifies the
employees by their response and their response and by their area of work. \\

\begin{tabular}{lll}
		      & Production & Non-Production \\
In Favour     & $129$      & $171$ \\
Not in Favour & $31$       & $69$ \\
\end{tabular} \\

Test whether there is evidence of a significant association between the response and the
area of work.

\textbf{Solution:}

\textbf{Null hypothesis:} \\
Two categorical variables are independent \\
$H_{0}$: Response independent of the area of work \\

\textbf{Alternative hypothesis:} \\
Two categorical variables are dependent \\
$H_{1}$: Response dependent of the area of work \\


\item[2.] The theory predicts that the proportion of beans in four given groups should be
$9:3:3:1$ in an examination with $1600$ beans, the numbers in the four groups were $882$, 
$313$, $287$ and $118$. Does the experiment result support the theory ? \\ 

\textbf{Solution:}

\end{enumerate}
\end{document}
