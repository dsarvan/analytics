\documentclass[a4paper,11pt,openright]{report}
\setlength{\parindent}{0pt} % set noindent for entire file

\usepackage[utf8]{inputenc}
\usepackage[a4paper, left=20mm, right=20mm, top=20mm]{geometry}
\usepackage{xcolor,graphicx}
\usepackage{amsmath}
\usepackage{setspace}
\usepackage{sectsty}
\usepackage{etoolbox}
\usepackage{enumitem}
\usepackage{listings}
\usepackage{times}

\graphicspath{ {/home/saran/Analytics/Apr_27/} }

\lstdefinestyle{mystyle}{
	backgroundcolor=\color{white},
	basicstyle=\ttfamily\footnotesize,
	breakatwhitespace=false,
	breaklines=true,
	captionpos=b,
	keepspaces=true,
	showspaces=false,
	showstringspaces=false,
	showtabs=false,
	tabsize=4
}

\lstset{style=mystyle}

\begin{document}
\singlespacing
\pagestyle{plain}

\begin{center}
\textbf{Answers to Question Set 6} \\
Date: 27/04/2020 \hspace{2mm} Name: D.Saravanan
\end{center}

\vspace{10px}

\textbf{Question 1:} \\
Write a python program to generate 9 random integers in numpy. Convert it to a 3$\times$3 matrix 
and then convert it to a dataframe. \\

Program: 
\lstinputlisting[language=Python]{question1.py}

\vspace{5px}

Output:
\lstinputlisting{output1.txt}

\vspace{10px}

\textbf{Question 2:} \\
Write a python program to generate a dataframe with the random values in column and 
calculate the mean and the standard deviation. \\

Program:
\lstinputlisting[language=Python]{question2.py}

\vspace{5px}

\pagebreak

Output:
\lstinputlisting{output2.txt}

\vspace{10px}

\textbf{Question 3:} \\
Write a python program to plot the mean and the standard deviation from the dataframe of
previous question using pandas inbuilt plot function and save the dataframe as JSON file 
format.\\

Program:
\lstinputlisting[language=Python]{question3.py}

\vspace{10px}

Output:

%figure_1
\begin{figure}[ht!]
\includegraphics[width=18cm,height=9cm,keepaspectratio]{dframe.pdf}
\end{figure}

\end{document}
